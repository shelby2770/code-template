\chapter{Combinatorial}

\section{Permutations}
	%\subsection{Factorial}
		%\import{factorial.tex}
		\kactlimport{IntPerm.h}
    \kactlimport{multinomial.h}
    \kactlimport{PrecomputedNCR.h}
    \kactlimport{Permutations.h}

\textbf{Cycles} Let $g_S(n)$ be the number of $n$-permutations whose cycle lengths all belong to the set $S$. Then $\sum_{n\geq 0} g_S(n) \frac{x^n}{n!} = \exp\left(\sum_{n\in S} \frac{x^n} {n} \right)$\\
\textbf{Derangements} Permutations of a set such that none of the elements appear in their original position. $\mkern-2mu D(n) = (n-1)(D(n-1)+D(n-2)) = n D(n-1)+(-1)^n = \left\lfloor\frac{n!}{e}\right\rceil$\\
\textbf{Burnside's Lemma} Given a group $G$ of symmetries and a set $X$, the number of elements of $X$ \emph{up to symmetry} equals ${\frac {1}{|G|}}\sum _{{g\in G}}|X^{g}|,$ where $X^{g}$ are the elements fixed by $g$ ($g.x = x$). If $f(n)$ counts ``configurations'' (of some sort) of length $n$, we can ignore rotational symmetry using $G = \mathbb Z_n$ to get $g(n) = \frac{1}{n} \sum_{k=0}^{n-1}{f(\text{gcd}(n, k))} = \frac{1}{n} \sum_{k|n}{f(k)\phi(n/k)}.$\\
\textbf{Partition function} Number of ways of writing $n$ as a sum of positive integers, disregarding the order of the summands. $p(0) = 1$, $p(n) = \sum_{k \in \mathbb Z \setminus \{0\}}{(-1)^{k+1} p(n - k(3k-1) / 2)}$. $p(n) \sim 0.145 / n \cdot \exp(2.56 \sqrt{n})$\\
First few values: 1, 1, 2, 3, 5, 7, 11, 15, 22, 30. $p(20) = 627, p(50) \approx 2e5, p(100) \approx 2e8$.\\
\textbf{Lucas' Theorem}: Let $n,m$ be non-negative integers and $p$ a prime. Write $n=n_kp^k+...+n_1p+n_0$ and $m=m_kp^k+...+m_1p+m_0$. Then $\binom{n}{m} \equiv \prod_{i=0}^k\binom{n_i}{m_i} \pmod{p}$.\\
\textbf{Bernoulli numbers} EGF of Bernoulli numbers is $B(t)=\frac{t}{e^t-1}$ (FFT-able). $\sum \frac{B_i}{i!}x^i = \frac{x}{1 - e^{-x}}$. $B[0,\ldots] = [1, -\frac{1}{2}, \frac{1}{6}, 0, -\frac{1}{30}, 0, \frac{1}{42}, \ldots]$.\\
Sums of powers: {\footnotesize \[ \sum_{i=1}^n n^m = \frac{1}{m+1} \sum_{k=0}^m \binom{m+1}{k} B_k \cdot (n+1)^{m+1-k} \] }\\
\textbf{Stirling numbers of the first kind} Number of permutations on $n$ items with $k$ cycles. $c(n,k) = c(n-1,k-1) + (n-1) c(n-1,k)$, $c(0,0) = 1$.$\sum_{k=0}^n c(n,k)x^k =$$x(x+1) \dots (x+n-1)$
{\footnotesize
$c(8,k) = 8, 0, 5040, 13068, 13132, 6769, 1960, 322, 28, 1$ \\
$c(n,2) = 0, 0, 1, 3, 11, 50, 274, 1764, 13068, 109584, ..$
}\\
\textbf{Stirling numbers of the second kind} Partitions of $n$ distinct elements into exactly $k$ non-empty subsets. $S(n,k) = S(n-1,k-1) + k S(n-1,k)$. $S(n,1) = S(n,n) = 1$. $S(n,k) = \frac{1}{k!}\sum_{j=0}^k (-1)^{k-j}\binom{k}{j}j^n$.\\
\textbf{Eulerian numbers} Number of $n$-permutations with exactly $k$ rises (positions $i$ with $p_i > p_{i-1}$). $E(n,k) = (n-k)E(n-1,k-1) + (k+1)E(n-1,k)$. $E(n,0) = E(n,n-1) = 1$. $E(n,k) = \sum_{j=0}^k(-1)^j\binom{n+1}{j}(k+1-j)^n$.\\
\textbf{Bell numbers} Total number of partitions of $n$ distinct elements. $B(n) =$ $1, 1, 2, 5, 15, 52, 203, 877, 4140, 21147, \dots$. $B(3) = 5 = \{a|b|c, a|bc, b|ac, c|ab, abc\}$. For $p$ prime, $B(p^m+n)\equiv mB(n)+B(n+1) \pmod{p}$.\\
\textbf{Catalan numbers}
$C_n=\frac{1}{n+1}\binom{2n}{n}= \binom{2n}{n}-\binom{2n}{n+1} = \frac{(2n)!}{(n+1)!n!}$\\
$C_0=1,\ C_{n+1} = \frac{2(2n+1)}{n+2}C_n,\ C_{n+1}=\sum C_iC_{n-i}$
${C_n = 1, 1, 2, 5, 14, 42, 132, 429, 1430, 4862, 16796, 58786, \dots}$
- UR path from $(0,0)$ to $(n,n)$ below $y=x$.\\
- strings with $n$ pairs of parenthesis, correctly nested.\\
- binary trees with with $n+1$ leaves (0 or 2 children).\\
- ordered trees with $n+1$ vertices.\\
- ways a convex polygon with $n+2$ sides can be cut into triangles by connecting vertices with straight lines.\\
- permutations of $[n]$ with no 3-term increasing subseq.\\
\textbf{Labeled unrooted trees}:
\# on $n$ vertices: $n^{n-2}$ \\
\# on $k$ existing trees of size $n_i$: $n_1n_2\cdots n_k n^{k-2}$ \\
\# with degrees $d_i$: $(n-2)! / ((d_1-1)! \cdots (d_n-1)!)$\\
\# ways to connect $k$ components with $k-1$ edges: $s_1 \cdots s_k \cdot n^{k-2}$\\
\textbf{Number of Spanning Trees} Create an $N\times N$ matrix \texttt{mat}, and for each edge $a \rightarrow b \in G$, do \texttt{mat[a][b]--, mat[b][b]++} (and \texttt{mat[b][a]--, mat[a][a]++} if $G$ is undirected). Remove the $i$th row and column and take the determinant; this yields the number of directed spanning trees rooted at $i$ (if $G$ is undirected, remove any row/column).\\
\textbf{Erdős–Gallai theorem} A simple graph with node degrees $d_1 \ge \dots \ge d_n$ exists iff $d_1 + \dots + d_n$ is even and for every $k = 1\dots n$,
\[ \sum _{i=1}^{k}d_{i}\leq k(k-1)+\sum _{i=k+1}^{n}\min(d_{i},k). \]
\textbf{Sprague-Grundy Theorem:} Viewing the game as a DAG, where a player moves from one node $v$ to any neighbor $v_i$, the grundy value $G(v) = mex\{v_i\}$ gives an equivalent pile of nim. If the game breaks into several equivalent games where player can move at any single part, take xorsum to combine (just like nim). Use DP/pattern hunting.\\
